\documentclass{article}
\usepackage{graphicx} 
\usepackage{subfigure}
\usepackage{mathtools}
\usepackage{amssymb}
\usepackage{chngcntr}
\begin{document}
\tableofcontents
\newpage
%\counterwithin*{equation}{section}
%\counterwithin*{equation}{subsection}
\section{Active Contours}
$E_{external}$ attracts the contour to the closest image edge.\\
$E_{internal}$ controls the contour to be continuous and smooth.
\begin{equation}\label{key}
E(c)=E_{internal}(c)+E_{external}(c)
\end{equation}
$$
E_{internal}(c)=\int_{0}^{1}\alpha||c'(s)||^2+\beta||c''(s)||^2ds
$$

$$
E_{external}(c)=\int_{0}^{1}-\gamma||\nabla I(c(s))||^2ds
$$
sometimes, noise can be existed in images, to have a better result, the method called Gradient Vector Field(GVF) is applied.\\
Drawbacks of active contours:\\
(1)has to keep points uniform;\\
(2)has to attract the contour to shrink to the object;\\
(3)can not surround multiple objects at one time;\\
(4)can not segment the inner hole.
\section{Level Sets}
Snake evolves a curve, while Level Sets evolves a function$\phi(x,y)$, makes it be 0.\\
Also called Region Based Active Contour Segmentation.
\subsection{Chan-Vese model}
\begin{equation}\label{key}
\begin{array}{ll}
E(c)=\mu(Length\ of\ C)+\nu(Area\ inside\ C)
\\\qquad\ +\lambda_1\int_{inside}(I(x,y)-\mu_{in})^2+\lambda_2\int_{outside}(I(x,y)-\mu_{out})^2
\end{array}
\end{equation}
\subsection{In terms of level set function $\varphi$}
\begin{multline}\label{key}
\arg\min_{c_1,c_2,\varphi} \ \mu\int_{\Omega}\delta(\varphi(x))|\nabla\varphi(x)|dx+\nu\int_{\Omega}H(\varphi(x))dx\\
+\lambda_1\int_{\Omega}|f(x)-c_1|^2H(\varphi(x))dx+\lambda_2\int_{\Omega}|f(x)-c_2|^2(1-H(\varphi(x)))dx
\end{multline}
\subsection{Euler-Lagrange equation}
\begin{multline}
\left\{ \begin{array}{ll}
\dfrac{\partial\varphi}{\partial t}=\delta(\varphi)[\mu div(\frac{\nabla\varphi}{|\nabla\varphi|})-\nu-\lambda_1(f-c_1)^2+\lambda_2(f-c_2)^2]\quad in\ \Omega\\
\dfrac{\delta(\varphi)}{|\nabla\varphi|}\dfrac{\partial\varphi}{\partial \vec{n}}=0\quad on\ \partial \Omega
\end{array}\right.
\end{multline}
\section{Graph Cuts}
Segmenting a image is actually adding labels in the each pixel. In graph cut algorithm, user gives the intial background and target which can be labeled as 0 and 1.\\
(1)Contruct a graph\\
A graph is made of 2 types of nodes and edges and weights.\\
Every pixel in the image is a node, another two nodes is Source and Terminal node;\\
The energy equation consists of Data item and Smooth item, respectively:
\begin{equation}\label{key}
E(L)=\lambda R(L)+B(L)
\end{equation}
Region part, which is actually the probability of p that belongs to background or object lable.\\
$$
R(L)=\sum_{p\in P}P_p(L_p)
$$
We want the energy equation as small as possible, so there is a ``-'' in front of probability.\\
The probability is calculated by comparing the gray value of pixel and greyscale histogram.
$$
R_p(1)=-lnPr(l_p|'obj')
$$
$$
R_p(0)=-lnPr(l_p|'bkg')
$$
Border item:\\
$$
B(L)=\sum_{p,q\in \mathcal{N}}B_{<p,q>}\cdot\delta(lp,lq)
$$
The more different the two neighboring pixels are, the smaller the Border item will be.
$$
B_{p,q}\propto\exp(-\dfrac{(I_p-I_q)^2}{2\sigma^2})
$$
Function $\delta(l_p,l_q)$ is actually a indicator function, that is to say, if the neighboring pixel has same known label, the border item doesn't need to be calculated.
$$
\delta(l_p,l_q)=\left\{ \begin{array}{ll}
0,\quad \mathrm{if} \{l_p=l_q\}\\
1,\quad \mathrm{if} \{l_p\neq l_q\}
\end{array}\right.
$$
(2)Cut a graph\\
Min cut(Max flow) is used to cut the graph, then the graph based image is segmented.
\section{Grabcut}
First, let's look at what are the differences between Graph Cuts and Grab Cut.\\
(1)Graph Cuts uses histogram as a model to calculate the probability of pixel belong to which label.\\
(2)Grab Cut uses GMM model based on RGB to calculate the probability of pixel belong to which label. So this model needs to iterate and learn to get better results.\\
(3)User has to draw bakground and object in Graph Cut, but one background frame is enough for Grab Cut.\\
\subsection{Model construct}
\begin{equation}\label{key}
E(\alpha,k,\theta,z)=U(\alpha,k,\theta,z)+V(\alpha,z)
\end{equation}
Data item:
$$
U(\alpha,k,\theta,z)=\sum_{n}D(\alpha_n,k_n,\theta,z_n)
$$
Smooth item:
$$
v(\alpha,z)=\gamma\sum_{(m,n)\in}C[\alpha_n\neq\alpha_m]\exp-\beta||z_m-z_n||^2
$$
\subsection{Iterative minimisation}
(1)Assign GMM components to pixels.\\
$$
k_n:=\arg\min_{k_n}D_n(\alpha_n,k_n,\theta,z_n)
$$
(2)Learn GMM parameters from data z.\\
$$
\theta=\{\pi(\alpha,k),\mu(\alpha,k),\sum(\alpha,k),\alpha=0,1,k=1\dots K\}
$$
$$
\theta:=\arg\min_{\theta} U(\alpha,k,\theta,z)
$$
(3)Use min cut(max flow) to segment.\\
(4)Repeat from step 1,until convergence.
\section{Optimization Techniques}
\end{document}